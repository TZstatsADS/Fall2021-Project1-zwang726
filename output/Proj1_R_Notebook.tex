% Options for packages loaded elsewhere
\PassOptionsToPackage{unicode}{hyperref}
\PassOptionsToPackage{hyphens}{url}
%
\documentclass[
]{article}
\usepackage{lmodern}
\usepackage{amssymb,amsmath}
\usepackage{ifxetex,ifluatex}
\ifnum 0\ifxetex 1\fi\ifluatex 1\fi=0 % if pdftex
  \usepackage[T1]{fontenc}
  \usepackage[utf8]{inputenc}
  \usepackage{textcomp} % provide euro and other symbols
\else % if luatex or xetex
  \usepackage{unicode-math}
  \defaultfontfeatures{Scale=MatchLowercase}
  \defaultfontfeatures[\rmfamily]{Ligatures=TeX,Scale=1}
\fi
% Use upquote if available, for straight quotes in verbatim environments
\IfFileExists{upquote.sty}{\usepackage{upquote}}{}
\IfFileExists{microtype.sty}{% use microtype if available
  \usepackage[]{microtype}
  \UseMicrotypeSet[protrusion]{basicmath} % disable protrusion for tt fonts
}{}
\makeatletter
\@ifundefined{KOMAClassName}{% if non-KOMA class
  \IfFileExists{parskip.sty}{%
    \usepackage{parskip}
  }{% else
    \setlength{\parindent}{0pt}
    \setlength{\parskip}{6pt plus 2pt minus 1pt}}
}{% if KOMA class
  \KOMAoptions{parskip=half}}
\makeatother
\usepackage{xcolor}
\IfFileExists{xurl.sty}{\usepackage{xurl}}{} % add URL line breaks if available
\IfFileExists{bookmark.sty}{\usepackage{bookmark}}{\usepackage{hyperref}}
\hypersetup{
  pdftitle={R Notebook},
  hidelinks,
  pdfcreator={LaTeX via pandoc}}
\urlstyle{same} % disable monospaced font for URLs
\usepackage[margin=1in]{geometry}
\usepackage{color}
\usepackage{fancyvrb}
\newcommand{\VerbBar}{|}
\newcommand{\VERB}{\Verb[commandchars=\\\{\}]}
\DefineVerbatimEnvironment{Highlighting}{Verbatim}{commandchars=\\\{\}}
% Add ',fontsize=\small' for more characters per line
\usepackage{framed}
\definecolor{shadecolor}{RGB}{248,248,248}
\newenvironment{Shaded}{\begin{snugshade}}{\end{snugshade}}
\newcommand{\AlertTok}[1]{\textcolor[rgb]{0.94,0.16,0.16}{#1}}
\newcommand{\AnnotationTok}[1]{\textcolor[rgb]{0.56,0.35,0.01}{\textbf{\textit{#1}}}}
\newcommand{\AttributeTok}[1]{\textcolor[rgb]{0.77,0.63,0.00}{#1}}
\newcommand{\BaseNTok}[1]{\textcolor[rgb]{0.00,0.00,0.81}{#1}}
\newcommand{\BuiltInTok}[1]{#1}
\newcommand{\CharTok}[1]{\textcolor[rgb]{0.31,0.60,0.02}{#1}}
\newcommand{\CommentTok}[1]{\textcolor[rgb]{0.56,0.35,0.01}{\textit{#1}}}
\newcommand{\CommentVarTok}[1]{\textcolor[rgb]{0.56,0.35,0.01}{\textbf{\textit{#1}}}}
\newcommand{\ConstantTok}[1]{\textcolor[rgb]{0.00,0.00,0.00}{#1}}
\newcommand{\ControlFlowTok}[1]{\textcolor[rgb]{0.13,0.29,0.53}{\textbf{#1}}}
\newcommand{\DataTypeTok}[1]{\textcolor[rgb]{0.13,0.29,0.53}{#1}}
\newcommand{\DecValTok}[1]{\textcolor[rgb]{0.00,0.00,0.81}{#1}}
\newcommand{\DocumentationTok}[1]{\textcolor[rgb]{0.56,0.35,0.01}{\textbf{\textit{#1}}}}
\newcommand{\ErrorTok}[1]{\textcolor[rgb]{0.64,0.00,0.00}{\textbf{#1}}}
\newcommand{\ExtensionTok}[1]{#1}
\newcommand{\FloatTok}[1]{\textcolor[rgb]{0.00,0.00,0.81}{#1}}
\newcommand{\FunctionTok}[1]{\textcolor[rgb]{0.00,0.00,0.00}{#1}}
\newcommand{\ImportTok}[1]{#1}
\newcommand{\InformationTok}[1]{\textcolor[rgb]{0.56,0.35,0.01}{\textbf{\textit{#1}}}}
\newcommand{\KeywordTok}[1]{\textcolor[rgb]{0.13,0.29,0.53}{\textbf{#1}}}
\newcommand{\NormalTok}[1]{#1}
\newcommand{\OperatorTok}[1]{\textcolor[rgb]{0.81,0.36,0.00}{\textbf{#1}}}
\newcommand{\OtherTok}[1]{\textcolor[rgb]{0.56,0.35,0.01}{#1}}
\newcommand{\PreprocessorTok}[1]{\textcolor[rgb]{0.56,0.35,0.01}{\textit{#1}}}
\newcommand{\RegionMarkerTok}[1]{#1}
\newcommand{\SpecialCharTok}[1]{\textcolor[rgb]{0.00,0.00,0.00}{#1}}
\newcommand{\SpecialStringTok}[1]{\textcolor[rgb]{0.31,0.60,0.02}{#1}}
\newcommand{\StringTok}[1]{\textcolor[rgb]{0.31,0.60,0.02}{#1}}
\newcommand{\VariableTok}[1]{\textcolor[rgb]{0.00,0.00,0.00}{#1}}
\newcommand{\VerbatimStringTok}[1]{\textcolor[rgb]{0.31,0.60,0.02}{#1}}
\newcommand{\WarningTok}[1]{\textcolor[rgb]{0.56,0.35,0.01}{\textbf{\textit{#1}}}}
\usepackage{graphicx}
\makeatletter
\def\maxwidth{\ifdim\Gin@nat@width>\linewidth\linewidth\else\Gin@nat@width\fi}
\def\maxheight{\ifdim\Gin@nat@height>\textheight\textheight\else\Gin@nat@height\fi}
\makeatother
% Scale images if necessary, so that they will not overflow the page
% margins by default, and it is still possible to overwrite the defaults
% using explicit options in \includegraphics[width, height, ...]{}
\setkeys{Gin}{width=\maxwidth,height=\maxheight,keepaspectratio}
% Set default figure placement to htbp
\makeatletter
\def\fps@figure{htbp}
\makeatother
\setlength{\emergencystretch}{3em} % prevent overfull lines
\providecommand{\tightlist}{%
  \setlength{\itemsep}{0pt}\setlength{\parskip}{0pt}}
\setcounter{secnumdepth}{-\maxdimen} % remove section numbering
\ifluatex
  \usepackage{selnolig}  % disable illegal ligatures
\fi

\title{R Notebook}
\author{}
\date{\vspace{-2.5em}}

\begin{document}
\maketitle

\hypertarget{this-report-aims-at-finding-out-what-are-the-different-schools-of-philosophy-usually-write.-the-data-used-is-the-dataset-for-philosophy-data-project.-it-contains-11-columns-and-360808-unique-values}{%
\paragraph{This report aims at finding out what are the different
schools of Philosophy usually write. The data used is the dataset for
Philosophy Data Project. It contains 11 columns and 360808 unique
values}\label{this-report-aims-at-finding-out-what-are-the-different-schools-of-philosophy-usually-write.-the-data-used-is-the-dataset-for-philosophy-data-project.-it-contains-11-columns-and-360808-unique-values}}

\begin{Shaded}
\begin{Highlighting}[]
\FunctionTok{setwd}\NormalTok{(}\StringTok{"\textasciitilde{}/Downloads"}\NormalTok{)}
\FunctionTok{library}\NormalTok{(tidyverse)}
\FunctionTok{library}\NormalTok{(dplyr)}
\FunctionTok{library}\NormalTok{(tm)}
\FunctionTok{library}\NormalTok{(wordcloud)}
\FunctionTok{library}\NormalTok{(RColorBrewer)}
\FunctionTok{library}\NormalTok{(beeswarm)}
\FunctionTok{library}\NormalTok{(sentimentr)}
\FunctionTok{library}\NormalTok{(tidytext)}
\FunctionTok{library}\NormalTok{(glue)}
\FunctionTok{library}\NormalTok{(syuzhet)}
\FunctionTok{library}\NormalTok{(caret)}
\FunctionTok{library}\NormalTok{(RTextTools)}
\NormalTok{df}\OtherTok{\textless{}{-}}\FunctionTok{read.csv}\NormalTok{(}\StringTok{"philosophy\_data.csv"}\NormalTok{)}
\end{Highlighting}
\end{Shaded}

\hypertarget{exploratory-data-analysis}{%
\subsection{Exploratory Data Analysis}\label{exploratory-data-analysis}}

First,I examined the data to see the columns and total number of rows,
and see the unique authors, schools and title, and then process the raw
textual data by cleaning data, removing stopwords and creating a tidy
version of texts which is saved in \$ output \$ file.

\begin{Shaded}
\begin{Highlighting}[]
\FunctionTok{dim}\NormalTok{(df)}
\end{Highlighting}
\end{Shaded}

\begin{verbatim}
## [1] 360808     11
\end{verbatim}

\begin{Shaded}
\begin{Highlighting}[]
\FunctionTok{colnames}\NormalTok{(df)}
\end{Highlighting}
\end{Shaded}

\begin{verbatim}
##  [1] "title"                     "author"                   
##  [3] "school"                    "sentence_spacy"           
##  [5] "sentence_str"              "original_publication_date"
##  [7] "corpus_edition_date"       "sentence_length"          
##  [9] "sentence_lowered"          "tokenized_txt"            
## [11] "lemmatized_str"
\end{verbatim}

\begin{Shaded}
\begin{Highlighting}[]
\FunctionTok{unique}\NormalTok{(df[}\FunctionTok{c}\NormalTok{(}\StringTok{\textquotesingle{}author\textquotesingle{}}\NormalTok{)])}
\end{Highlighting}
\end{Shaded}

\begin{verbatim}
##                 author
## 1                Plato
## 38367        Aristotle
## 87146            Locke
## 96031             Hume
## 104343        Berkeley
## 107077         Spinoza
## 110870         Leibniz
## 115897       Descartes
## 117029     Malebranche
## 130026         Russell
## 135099           Moore
## 138767    Wittgenstein
## 145817           Lewis
## 158937           Quine
## 166310          Popper
## 170988          Kripke
## 185451        Foucault
## 200691         Derrida
## 206690         Deleuze
## 219230   Merleau-Ponty
## 226822         Husserl
## 232564       Heidegger
## 247803            Kant
## 261931          Fichte
## 267239           Hegel
## 289939            Marx
## 303428           Lenin
## 307897           Smith
## 319590         Ricardo
## 322680          Keynes
## 326091       Epictetus
## 326414 Marcus Aurelius
## 328626       Nietzsche
## 342174  Wollstonecraft
## 344733        Beauvoir
## 357750           Davis
\end{verbatim}

\begin{Shaded}
\begin{Highlighting}[]
\FunctionTok{unique}\NormalTok{(df[}\FunctionTok{c}\NormalTok{(}\StringTok{\textquotesingle{}school\textquotesingle{}}\NormalTok{)])}
\end{Highlighting}
\end{Shaded}

\begin{verbatim}
##                 school
## 1                plato
## 38367        aristotle
## 87146       empiricism
## 107077     rationalism
## 130026        analytic
## 185451     continental
## 219230   phenomenology
## 247803 german_idealism
## 289939       communism
## 307897      capitalism
## 326091        stoicism
## 328626       nietzsche
## 342174        feminism
\end{verbatim}

\begin{Shaded}
\begin{Highlighting}[]
\FunctionTok{unique}\NormalTok{(df[}\FunctionTok{c}\NormalTok{(}\StringTok{\textquotesingle{}title\textquotesingle{}}\NormalTok{)])}
\end{Highlighting}
\end{Shaded}

\begin{verbatim}
##                                                          title
## 1                                       Plato - Complete Works
## 38367                               Aristotle - Complete Works
## 87146                            Second Treatise On Government
## 88289                     Essay Concerning Human Understanding
## 96031                               A Treatise Of Human Nature
## 103078                   Dialogues Concerning Natural Religion
## 104343                                         Three Dialogues
## 106037 A Treatise Concerning The Principles Of Human Knowledge
## 107077                                                  Ethics
## 110381                     On The Improvement Of Understanding
## 110870                                                Theodicy
## 115897                                     Discourse On Method
## 116237                         Meditations On First Philosophy
## 117029                                  The Search After Truth
## 130026                                    The Analysis Of Mind
## 133539                              The Problems Of Philosophy
## 135099                                   Philosophical Studies
## 138767                            Philosophical Investigations
## 144605                          Tractatus Logico-Philosophicus
## 145817                                          Lewis - Papers
## 158937                                            Quintessence
## 166310                       The Logic Of Scientific Discovery
## 170988                                    Naming And Necessity
## 173669                                  Philosophical Troubles
## 183467                                            On Certainty
## 185451                                 The Birth Of The Clinic
## 187969                                      History Of Madness
## 196002                                     The Order Of Things
## 200691                                  Writing And Difference
## 206690                               Difference And Repetition
## 212551                                            Anti-Oedipus
## 219230                         The Phenomenology Of Perception
## 226822   The Crisis Of The European Sciences And Phenomenology
## 231654                               The Idea Of Phenomenology
## 232564                                          Being And Time
## 241069                                    Off The Beaten Track
## 247803                            Critique Of Practical Reason
## 250255                                   Critique Of Judgement
## 254459                                 Critique Of Pure Reason
## 261931                                    The System Of Ethics
## 267239                                        Science Of Logic
## 277917                             The Phenomenology Of Spirit
## 285016                     Elements Of The Philosophy Of Right
## 289939                                                 Capital
## 302935                                 The Communist Manifesto
## 303428                                Essential Works Of Lenin
## 307897                                   The Wealth Of Nations
## 319590     On The Principles Of Political Economy And Taxation
## 322680     A General Theory Of Employment, Interest, And Money
## 326091                                             Enchiridion
## 326414                                             Meditations
## 328626                                          The Antichrist
## 329796                                    Beyond Good And Evil
## 331702                                               Ecce Homo
## 333206                                   Twilight Of The Idols
## 336258                                  Thus Spake Zarathustra
## 342174                      Vindication Of The Rights Of Woman
## 344733                                          The Second Sex
## 357750                                  Women, Race, And Class
\end{verbatim}

Visualize the data group by school to get a sense of how many values of
each school in contained in this dataset. We can see that analytic,
aristotle, germain\_idealism, and plato has the most counts among the 11
schools.
\includegraphics{Proj1_R_Notebook_files/figure-latex/unnamed-chunk-4-1.pdf}

To use the tm package we first transfrom the dataset to a corpus and
next we normalize the texts in the sentence\_lowered using a series of
pre-processing steps: 1. Remove numbers 2. Remove punctuation marks and
stopwords 3. Remove extra whitespaces. After the above transformations
the first review looks like:

\begin{verbatim}
## <<SimpleCorpus>>
## Metadata:  corpus specific: 1, document level (indexed): 0
## Content:  documents: 1
## 
## [1]  whats new socrates make leave usual haunts lyceum spend time king archons court
\end{verbatim}

To analyze the textual data, we use a Document-Term Matrix (DTM)
representation.To reduce the dimension of the DTM, we can emove the less
frequent terms such that the sparsity is less than 0.99.

\begin{verbatim}
## <<DocumentTermMatrix (documents: 360808, terms: 94453)>>
## Non-/sparse entries: 4092757/34075305267
## Sparsity           : 100%
## Maximal term length: 58
## Weighting          : term frequency (tf)
\end{verbatim}

\begin{verbatim}
## <<DocumentTermMatrix (documents: 6, terms: 6)>>
## Non-/sparse entries: 0/36
## Sparsity           : 100%
## Maximal term length: 8
## Weighting          : term frequency (tf)
## Sample             :
##      Terms
## Docs  ventures main wiser became denounce found
##   500        0    0     0      0        0     0
##   501        0    0     0      0        0     0
##   502        0    0     0      0        0     0
##   503        0    0     0      0        0     0
##   504        0    0     0      0        0     0
##   505        0    0     0      0        0     0
\end{verbatim}

\begin{verbatim}
## <<DocumentTermMatrix (documents: 360808, terms: 152)>>
## Non-/sparse entries: 1081776/53761040
## Sparsity           : 98%
## Maximal term length: 13
## Weighting          : term frequency (tf)
\end{verbatim}

The first review now looks like:

\begin{verbatim}
## <<DocumentTermMatrix (documents: 1, terms: 20)>>
## Non-/sparse entries: 3/17
## Sparsity           : 85%
## Maximal term length: 9
## Weighting          : term frequency (tf)
## Sample             :
##     Terms
## Docs called else know long make much must new say time
##    1      0    0    0    0    1    0    0   1   0    1
\end{verbatim}

Inspect the frequent terms:

\begin{verbatim}
##   [1] "make"          "new"           "time"          "say"          
##   [5] "else"          "must"          "know"          "called"       
##   [9] "long"          "much"          "rather"        "one"          
##  [13] "think"         "knowledge"     "man"           "subject"      
##  [17] "thing"         "men"           "first"         "good"         
##  [21] "just"          "others"        "possible"      "right"        
##  [25] "take"          "way"           "become"        "great"        
##  [29] "seems"         "will"          "may"           "true"         
##  [33] "things"        "makes"         "case"          "nothing"      
##  [37] "yet"           "like"          "matter"        "mind"         
##  [41] "perhaps"       "reason"        "whether"       "anything"     
##  [45] "even"          "people"        "without"       "clear"        
##  [49] "come"          "thought"       "far"           "part"         
##  [53] "difference"    "let"           "hand"          "since"        
##  [57] "ideas"         "use"           "indeed"        "now"          
##  [61] "shall"         "find"          "well"          "every"        
##  [65] "everything"    "kind"          "form"          "can"          
##  [69] "law"           "already"       "said"          "however"      
##  [73] "god"           "many"          "truth"         "two"          
##  [77] "upon"          "mean"          "also"          "state"        
##  [81] "cause"         "according"     "another"       "different"    
##  [85] "question"      "present"       "never"         "point"        
##  [89] "experience"    "found"         "give"          "though"       
##  [93] "nature"        "neither"       "merely"        "something"    
##  [97] "means"         "contrary"      "either"        "fact"         
## [101] "general"       "principle"     "see"           "place"        
## [105] "made"          "together"      "number"        "always"       
## [109] "parts"         "order"         "still"         "example"      
## [113] "sense"         "object"        "words"         "given"        
## [117] "life"          "work"          "real"          "understanding"
## [121] "whole"         "power"         "particular"    "might"        
## [125] "language"      "therefore"     "latter"        "human"        
## [129] "common"        "certain"       "existence"     "meaning"      
## [133] "view"          "soul"          "body"          "less"         
## [137] "value"         "idea"          "end"           "necessary"    
## [141] "thus"          "within"        "pure"          "objects"      
## [145] "natural"       "relation"      "world"         "self"         
## [149] "universal"     "concept"       "consciousness" "labour"
\end{verbatim}

We can draw a wordcloud:
\includegraphics{Proj1_R_Notebook_files/figure-latex/unnamed-chunk-10-1.pdf}

One may argue that in the wordcloud, some words do not carry too much
meaning in the setting. Therefore we use tf--idf instead of the
frequencies of the term as entries, tf-idf measures the relative
importance of a word to a document.

\begin{verbatim}
## <<DocumentTermMatrix (documents: 360808, terms: 4)>>
## Non-/sparse entries: 104660/1338572
## Sparsity           : 93%
## Maximal term length: 4
## Weighting          : term frequency - inverse document frequency (normalized) (tf-idf)
\end{verbatim}

The new wordcloud is more informative, only gives us 4 words.
\includegraphics{Proj1_R_Notebook_files/figure-latex/unnamed-chunk-12-1.pdf}

We then try to see the average sentence length group by school and the
overall distribution of sentence length. We can see that Capitalism and
Empiricism has the longest sentence length while Plato, Analytic, and
Nietzsche has the shortest.
\includegraphics{Proj1_R_Notebook_files/figure-latex/unnamed-chunk-13-1.pdf}
\includegraphics{Proj1_R_Notebook_files/figure-latex/unnamed-chunk-13-2.pdf}

We plot violin plots of sentence length group by school and author to
see the distribution.
\includegraphics{Proj1_R_Notebook_files/figure-latex/unnamed-chunk-14-1.pdf}

\includegraphics{Proj1_R_Notebook_files/figure-latex/unnamed-chunk-15-1.pdf}

\hypertarget{sentiment-analysis}{%
\subsection{Sentiment Analysis}\label{sentiment-analysis}}

In this section we break down the data set by school and create
wordlcouds based on tf-idf criteria and perform sentiment analysis to
see the overall sentiment of this school.

Plato

\begin{verbatim}
## # A tibble: 1 x 3
##   negative positive sentiment
##      <dbl>    <dbl>     <dbl>
## 1    21186    32083     10897
\end{verbatim}

\includegraphics{Proj1_R_Notebook_files/figure-latex/unnamed-chunk-16-1.pdf}

Aristotle

\begin{verbatim}
## # A tibble: 1 x 3
##   negative positive sentiment
##      <dbl>    <dbl>     <dbl>
## 1    29979    35453      5474
\end{verbatim}

\includegraphics{Proj1_R_Notebook_files/figure-latex/unnamed-chunk-17-1.pdf}

Empiricism

\begin{verbatim}
## # A tibble: 1 x 3
##   negative positive sentiment
##      <dbl>    <dbl>     <dbl>
## 1    17213    20541      3328
\end{verbatim}

\includegraphics{Proj1_R_Notebook_files/figure-latex/unnamed-chunk-18-1.pdf}

Rationalism

\begin{verbatim}
## # A tibble: 1 x 3
##   negative positive sentiment
##      <dbl>    <dbl>     <dbl>
## 1    18770    24852      6082
\end{verbatim}

\includegraphics{Proj1_R_Notebook_files/figure-latex/unnamed-chunk-19-1.pdf}

Analytic

\begin{verbatim}
## # A tibble: 1 x 3
##   negative positive sentiment
##      <dbl>    <dbl>     <dbl>
## 1    23880    23821       -59
\end{verbatim}

\includegraphics{Proj1_R_Notebook_files/figure-latex/unnamed-chunk-20-1.pdf}

Continental

\begin{verbatim}
## # A tibble: 1 x 3
##   negative positive sentiment
##      <dbl>    <dbl>     <dbl>
## 1    33451    22008    -11443
\end{verbatim}

\includegraphics{Proj1_R_Notebook_files/figure-latex/unnamed-chunk-21-1.pdf}

Phenomenology

\begin{verbatim}
## # A tibble: 1 x 3
##   negative positive sentiment
##      <dbl>    <dbl>     <dbl>
## 1    13338    14552      1214
\end{verbatim}

\includegraphics{Proj1_R_Notebook_files/figure-latex/unnamed-chunk-22-1.pdf}

German\_idealism

\begin{verbatim}
## # A tibble: 1 x 3
##   negative positive sentiment
##      <dbl>    <dbl>     <dbl>
## 1    25103    33830      8727
\end{verbatim}

\includegraphics{Proj1_R_Notebook_files/figure-latex/unnamed-chunk-23-1.pdf}

Communism

\begin{verbatim}
## # A tibble: 1 x 3
##   negative positive sentiment
##      <dbl>    <dbl>     <dbl>
## 1    10390    10934       544
\end{verbatim}

\includegraphics{Proj1_R_Notebook_files/figure-latex/unnamed-chunk-24-1.pdf}

Capitalism

\begin{verbatim}
## # A tibble: 1 x 3
##   negative positive sentiment
##      <dbl>    <dbl>     <dbl>
## 1    11440    17491      6051
\end{verbatim}

\includegraphics{Proj1_R_Notebook_files/figure-latex/unnamed-chunk-25-1.pdf}

Nietzsche

\begin{verbatim}
## # A tibble: 1 x 3
##   negative positive sentiment
##      <dbl>    <dbl>     <dbl>
## 1    12445    12580       135
\end{verbatim}

\includegraphics{Proj1_R_Notebook_files/figure-latex/unnamed-chunk-26-1.pdf}

Stoicism

\begin{verbatim}
## # A tibble: 1 x 3
##   negative positive sentiment
##      <dbl>    <dbl>     <dbl>
## 1     2078     2342       264
\end{verbatim}

\includegraphics{Proj1_R_Notebook_files/figure-latex/unnamed-chunk-27-1.pdf}

Feminism

\begin{verbatim}
## # A tibble: 1 x 3
##   negative positive sentiment
##      <dbl>    <dbl>     <dbl>
## 1    20311    20438       127
\end{verbatim}

\includegraphics{Proj1_R_Notebook_files/figure-latex/unnamed-chunk-28-1.pdf}

\end{document}
